\documentclass[usejistfm]{ieej}
\usepackage{url}
\usepackage{amsmath}
\usepackage[dvipdfmx]{graphicx}
\usepackage[dvipdfmx]{color}
\usepackage[varg]{txfonts}
\usepackage{bm}
\usepackage[caption=false]{subfig}

%\usepackage[top=0.5in]{geometry}
%\usepackage{fullpage}
%\usepackage{auto-pst-pdf}
%\usepackage{epstopdf}
%\usepackage{float}
\usepackage{subfloat}
%\usepackage{caption}
%\usepackage{axodraw4j}
%\usepackage{pstricks}


\jtitle{半導体歪ゲージを用いたハイダイナミックレンジ1軸力覚センサの開発}
%\etitle{Development of a compact high dynamic range 6-axis force/torque sensor}
\authorlist{
 \authorentry{19MM227\ \ \ 田村 龍也}{Ryuya Tamura}{s}{TRL}
 }
%\affiliate[] {埼玉大学工学部電気電子システム工学科\\
 %〒338--8570\ 埼玉県さいたま市下大久保255}
  %{Saitama\ University \\
  %255, Shimo-ohkubo, Sakura-ku\ Saitama\ 338--8570}
\begin{document}
\begin{abstract}

%\input{./src/Abstract}

\end{abstract}
%\begin{jkeyword}
 %力覚センサ,ダイナミックレンジ, 小型化, 分解能
%\end{jkeyword}
%\begin{ekeyword}
 %Force/Torque sensor, Dynamic range, Miniaturization, Resolution
%\end{ekeyword}
\maketitle


\section{まえがき} 
ロボットの活躍領域は近年急速に拡大しており, 
それとともに普及率も上昇している. 
従来は自動車や家電など製造を行う産業用途での活躍が主であったが, 
現在は医療, 介護やサービス業などの第三次産業に有効性が見込まれ, 
協働ロボット等が多く開発されていることからもその背景がうかがい知れる.

協働ロボットの中でも特に人間を支援するまたは人間の代替として活動するロボットが注目されている. 
これらのロボットには環境適応力が求められ, 環境情報の取得が必要不可欠となっている.
中でも力覚情報は多様な動作の実現のためには必須だと考えられ, 
実際ロットによる多くの作業で力覚センサが用いられている\cite{denso}\cite{asimo}\cite{ROBEAR}.

これまでの力覚センサは, 発生する力の大きさに合わせて, 測定に際し
適切なレンジの力覚センサを選出し使用されてきた. 
しかし多種多様な動作を一台のロボットで実現させる場合, それに導入する力覚センサは
レンジが限定的なものではなく, よりハイダイナミックレンジ(HDR:High Dynamic Range)で
あることが求められる. 
また, HDRなだけではなく, ロボットに取り付け可能なサイズであることが求められる. 

このような需要に対し, Jiangらは低剛性起歪体と高剛性起歪体を組み合わせた1軸\cite{Jiang2015}, 2軸\cite{jiang2013}
の力情報を検知可能なHDR力覚センサを提案した. 
また起歪体を用いず, 水晶振動子による圧電効果を利用しHDRで力覚検知を可能にした
1軸のセンサも提案されている\cite{murozaki2014miniaturized}.

さらに1軸, 2軸での検知のみであったHDR力覚センサに対し, Okumuraらは6軸の力情報を取得可能な
HDR力覚センサ(size:150×150×45 mm)を提案した\cite{okumura2018high} \cite{Okumura}. 
これはそれぞれが6軸の力情報を取得できる低剛性起歪体と高剛性起歪体を二段に重ね合わせた構造をしており, 
0.01Nから1000Nまでの力覚検知が可能である. 従来の10倍以上のダイナミックレンジを
有した力覚センサとなった. 

%力覚センサは外力によって生じる構造体(以下, 起歪体)の変形量をセンシング素子により検知し,
%その変形量を元に外力の推定を行なう. 
%センシング素子により力覚センサの測定方式は異なり, 
%ひずみゲージ式\cite{yoshikawa1989six}%\cite{nishiwaki2002six}\cite{Liang2010}, 
%静電容量式\cite{Beyeler2009}, 
%光学式\cite{Kim2013a}%\cite{su20093}\cite{polygerinos2010novel}
%など様々な検出方法が実用化されている. 

%基本的に力覚センサの測定レンジは起歪体の剛性により決まり, 
%低剛性の起歪体は小さな力の検出が, 高剛性の起歪体は大きな力の検出が可能である. 
%しかしこれは反対に, 1つの起歪体だけでは小さな力と大きな力の検出を行なえないことを示唆している. 
%起歪体の変形を検知するという原理上, 分解能と定格荷重はトレードオフの関係にあり, 従来の力覚センサの
%ダイナミックレンジ(測定可能な力の最小値と最大値の比)には制限が生じる. 

しかし, 提案されてきたHDR力覚センサはセンサ自体のサイズが大きく, 
先述したようなロボットに導入する上で大きな問題となる.

これに対し我々は起歪体の構造自体を工夫することでHDRと小型化の両立を図るセンサ\cite{Ryuya}の提案をした. 
このセンサは低剛性起歪体の形状がクロスアーチ型となっており, 
高剛性起歪体の隙間にフィットする構造となっている. 
センササイズは(80×80×23.5 mm)と小型化に成功し, 
測定レンジは0.2N~500Nを確保出来た.
しかし, 0.2N以下の低荷重域での測定が不向きであることや, 
従来センサのHDRの広さを保つことが出来なかった. 

このことから, 小型かつHDRを有した力覚センサの実現には, 
起歪体の構造を工夫するのではなく, 
センシング方法の新しい提案が必要であると考えた. 

そこで本論文では, HDRを実現する新たなる手法として, 半導体ひずみゲージと
金属箔ひずみゲージを併用した小型で単純な構造の力覚センサを提案する. 
従来は剛性の異なる起歪体を多段に使用することで実現していたHDRを, 
本センサでは単純な起歪体構造(片持ち梁)上に感度の異なるひずみゲージを導入すること(のみ)でHDRを実現した. 

%以下に本論文の構成を示す. まず,2章で力覚センサに関連する原理を述べる.
%次に3章では提案する小型HDR6軸力覚センサを示し,4章でシミュレーションによって, 提案する力覚センサの
%挙動と設計の妥当性を確認する. 5章では実際に製作した力覚センサの有用性を確認するため行なった性能試験の結果を述べ, 
%最後の6章でまとめとする. 

\section{力覚センサの原理}
\subsection{歪ゲージ}
材料に引張力(または圧縮力)が加わる時, これに対応する応力が材料内部に発生する.
ここでこの応力に比例した引張歪(または圧縮歪)が発生し, 長さLの材料は  に変形する.
この時の と の割合を歪と呼ぶ. 

歪ゲージはこの歪を電気信号として検出することのできるセンシング素子のことである. 


\subsection{金属箔歪ゲージ}*
\subsection{半導体歪ゲージ}*

\subsection{起歪体}

%6軸力覚センサとはデカルト座標系における$x, y, z$軸方向の力($Fx, Fy, Fz$)と力のモーメント($Mx, My, Mz$)の大きさを
%測定するセンサである. 本研究では一般的に広く利用されているひずみゲージ式を採用した. 

%力覚センサは起歪体に生じるひずみの変化量を力情報へと変換する. 
%力覚センサに印加される荷重\textbf{\textit{L}}($F$[N], $M$[Nm])と
%ひずみ出力\textbf{\textit{S}}[$\mu$m/m]は次のように表せる. 
%\begin{eqnarray}
%  \bm{L} = {[F_x, F_y, F_z, M_x, M_y, M_z]}^{\top} \\
%  \bm{S} = {[ S_1, S_2, S_3, S_4, S_5, S_6 ]}^{\top}
%\end{eqnarray}
%荷重$\bm{L}$とひずみ出力$\bm{S}$は荷重-ひずみ行列$\bm{C}$によって次のように関係付けられる. 
%\begin{eqnarray}
%  \bm{S} = \bm{C}\bm{L}
%\end{eqnarray}
%また, 荷重-ひずみ行列$\bm{C}$の逆行列を用いると
%\begin{eqnarray}
%  \bm{L} = \bm{C}^{-1}\bm{S}
%  \label{eq:syuturyoku}
%\end{eqnarray}
%となる. よってひずみ出力$\bm{S}$を元に印加された荷重$\bm{L}$の計測が可能となる. 

%次に較正行列の求め方\cite{hanyu2010simplified}を述べる. ここで, すでに印加荷重$\bm{L}$と
%そのときのひずみ$\bm{S}$の値が既知であり, そのサンプル数は$n$であるとする.
%つまり, $\bm{L} = [\bm{L}^1, \bm{L}^2, \cdots , \bm{L}^n]$, $\bm{S} = [\bm{S}^1, \bm{S}^2, \cdots , \bm{S}^n]$
%が既に与えられているとする。ここで、$\bm{L}^k = [F^k_x, F^k_y, F^k_z, M^k_x, M^k_y, M^k_z]^{\top}$, $\bm{S}^k = [S^k_1, S^k_2, \cdots, S^k_i]^{\top} $
%とすると
%\begin{eqnarray}
%  \bm{C}^{-1} = \bm{L} \bm{S}^{-1} \label{eq:c}
%\end{eqnarray}
%によって$\bm{C}^{-1}$を求めることが可能である. 
%以下、簡単のため$\bm{L}$の$F_x$のみを考える。較正行列$\bm{C}$の逆行列$\bm{C}^{-1}$を
%\begin{eqnarray}
%\bm{C}^{-1} = \left[
%   \begin{array}{cccc}
%     a_{11} & a_{12} & \ldots & a_{16} \\
%     a_{21} & a_{22} & \ldots & a_{26} \\
%     \vdots & \vdots & \ddots & \vdots \\
%     a_{61} & a_{62} & \ldots & a_{66}
%   \end{array}
% \right]
%\end{eqnarray}
%とすると、$k$サンプル目の$F^k_x$の予測値$F^{k\ast}_x$は
%\begin{eqnarray}
%   F^{k\ast}_x = a^k_{11} S^k_1 + a^k_{12} S^k_2 + \cdots + a^k_{16} S^k_6
%\end{eqnarray}
%と表される。
%式\eqref{eq:c}は実際の印加荷重と予測値の残差二乗和
%\begin{eqnarray}
%  \sum_{k=0}^{n} d^2_k = \sum_{k=0}^{n} {\left(F^k_x - F^{k\ast}_x \right)}^2
%\end{eqnarray}
%が最小となるような定数$[a_{11}, a_{12}, \ldots, a_{16}]$を最小二乗法で導出するのと等価である. 
%ことで、$\bm{C}^{-1}$の$F_x$に寄与する部分を求めることができる。他の軸も同様に求めることができ、合わせることで較正行列$\bm{C}$の逆行列$\bm{C}^{-1}$を求めることができる。


\section{HDR1軸力覚センサの設計}
\subsection{起歪体の構造}
提案する力覚センサの構造をFig.~\ref{fig:sensor}に示す. 

従来の多段型力覚センサは低剛性起歪体と高剛性起歪体とが上下に重なる構造をしており, 
高さ方向にサイズがかさばるといった問題が生じていた. 
今回提案するクロスアーチ型の低剛性起歪体は, 中央の平板の各辺から梁が水平に伸び, 
ある点から折り曲がった構造をしている. 
これにより梁のない空間が必ず生まれる特徴がある. 
このクロスアーチ型を導入したことで, 低剛性起歪体の梁のない空間に高剛性起歪体を収めることができ, 
高さ方向のサイズの軽減が可能となった. 
また, 従来の樹脂製HDR6軸力覚センサは各起歪体を水平に配置しているため, 梁の長さを確保すると
長さ, 幅方向にサイズが大きくなってしまっていた. 本力覚センサは提案構造により, 
異なる高さで梁の長さを確保できるようになる. よって長さ, 幅方向のサイズの軽減も可能となる.

力覚センサは小さい力を
検知するための低剛性起歪体と大きい力を検知するための高剛性起歪体および
過負荷防止機構によって構成される. 
力覚センサに印加された荷重は, 低剛性起歪体の測定レンジ内では2つの起歪体に伝達される. 
低剛性起歪体の測定レンジ外の荷重は過負荷防止機構が作動することにより, 
低剛性起歪体にかかる負荷は一定となり, 高剛性起歪体が伝達された荷重によって大きくひずむ. 
過負荷防止機構により低剛性起歪体の過負荷防止と, 
印加荷重によって負荷が伝達される起歪体を決めることが可能となっている. 
\begin{figure}[b]
  \begin{center}
    \includegraphics[width=6.0cm]{pic/sensor.png}
    \caption{全体の構造}\label{fig:sensor}
  \end{center}
 \end{figure}
 \begin{figure}[b]
  \centering
  \subfloat[低剛性起歪体]{\includegraphics[width=3.5cm]{pic/LowCAD.png}}
  \subfloat[高剛性起歪体]{\includegraphics[width=3.5cm]{pic/HighCAD.png}}\\
  \caption[]{各起歪体の構造}\label{fig:kiwaitai}
\end{figure}
%\clearpage

\subsection{梁の設計}
力覚センサの中でも梁の構造はセンサ性能に大きく影響する重要な要素であり, 
その構造は任意の定格荷重, 荷重印加時のひずみ, 安全率といったパラメータを参考に決定される.
よって有限要素法シミュレーションにより荷重印加時の起歪体の挙動を調べた.

各起歪体に力と力のモーメント, 併せて6成分の定格荷重を印加した. 
この時の安全率をTable~\ref{tb:anzen}に示す. 
またFig.~\ref{fig:sim}に$-F_z$方向に力を加えた時の各起歪体の挙動を示す.
本力覚センサで設定した定格荷重はTable~\ref{tb:kajuu}に示す. 
\begin{table}[h]
  \caption{定格荷重($F$[N], $M$[Nm])}\label{tb:kajuu}
  \begin{center}
   \begin{tabular}{ c c c c c c c }
    \hline
     & $F_x$ & $F_y$ & $F_z$ & $M_x$ & $M_y$ & $M_z$  \\
    \hline
    Low-rigidity & 50 & 50 & 50 & 0.6 & 0.6 & 0.6  \\
    \hline
    High-rigidity & 500 & 500 & 500 & 10 & 10 & 12  \\
    \hline   
   \end{tabular}
  \end{center}
 \end{table}
\begin{table}[h]
  \caption{安全率\label{tb:anzen}}
  \begin{center}
   \begin{tabular}{ c c c c c c c }
    \hline
     & $F_x$ & $F_y$ & $F_z$ & $M_x$ & $M_y$ & $M_z$  \\
    \hline
    Low-rigidity & 2.04 & 2.04 & 1.83 & 1.78 & 1.78 & 3.13  \\
    \hline
    High-rigidity & 1.37 & 1.37 & 1.39 & 1.23 & 1.23 & 1.78  \\
    \hline   
   \end{tabular}
  \end{center}
 \end{table}
\subsection*{低剛性起歪体}
梁は低荷重に対して高感度である必要がある. よって, より細く長い梁が望ましい. 
しかし定格荷重を印加した時に塑性変形してしまう細さではいけない. 本力覚センサの低剛性起歪体の
梁の太さは2mm角に設計した. Fig.~\ref{fig:sim}~(a)より印加された荷重に対し
良好なひずみが得られたことがわかる. さらに, 6成分の定格荷重に対して
安全率が1以上を確保できている.
よって設計の妥当性が示された. 

\subsection*{高剛性起歪体}
高剛性起歪体の構造はクロスビーム型を参考としており,
外側剛体, 中央剛体, 弾性梁, 弾性薄板から構成される. 
弾性梁は5mm角で22.5mmの長さ, 弾性薄板は2mm厚, 9mm幅で56mmの長さである. 
Fig.~\ref{fig:sim}~(b)より印加された荷重に対し良好なひずみが得られたことがわかる. 
さらに, 6成分の定格荷重に対して安全率が1以上を確保できている. 
よって設計の妥当性が示された. 

設計した各起歪体の寸法をTable~\ref{tb:size} に示す. 
\begin{table}[h]
  \caption{起歪体の寸法[mm]}\label{tb:size}
  \begin{center}
   \begin{tabular}{ c c c c c }
    \hline
     & & Length & width & Height  \\
    \hline
    Low-rigidity & Body & 80 & 80 & 22  \\
    \cline{2-5}
     & Elastic bearn & 28 & 2 & 2  \\
    \hline
    High-rigidity & Body & 80 & 80 & 23.5  \\
    \cline{2-5}
     & Elastic bearn & 22.5 & 5 & 5  \\
    \cline{2-5}
     & Thin plate & 56 & 2 & 9 \\
    \hline
   \end{tabular}
  \end{center}
 \end{table}

\subsection{過負荷防止機構の構造}
過負荷防止機構の構造をFig.~\ref{fig:kahuka}に示す. 
過負荷防止機構はストッパピン, ストッパ板, 外側剛体(高剛性起歪体)から構成されている. 
6軸方向の各荷重に対し, 過負荷防止機構は以下のように動作する. 過負荷防止機構の作動荷重は
ストッパピン, ストッパ板, 外側剛体それぞれとのクリアランスによって決定される. 
本力覚センサのクリアランスの寸法をTable~\ref{tb:clea}に示す. 
\\
\begin{itemize}
  \item $\pm F_x$および$\pm F_y$:ストッパピンとストッパ板が接触. 
  \item $+F_z$:ストッパピンとストッパ板が接触. 
  \item $-F_z$:ストッパ板と外側剛体が接触. 
  \item $\pm M_x$および$\pm M_y$:ストッパ板がストッパピン, 外側剛体と接触
  \item $\pm M_z$:ストッパピンとストッパ板が接触. 
\end{itemize}
\begin{table}[h]
  \caption{過負荷防止機構のクリアランス[mm]}\label{tb:clea}
  \begin{center}
   \begin{tabular}{ c c }
    \hline
    Vertical clearance & Horizontal clearance  \\
    \hline
    0.3 & 0.3  \\
    \hline
   \end{tabular}
  \end{center}
 \end{table}

\subsection{ひずみゲージ貼付位置}\label{sec:hizumiharituke}
Fig.~\ref{fig:gage}に低剛性, 高剛性起歪体それぞれのひずみゲージの貼付位置を示す. 

ここで, 各起歪体にはそれぞれ$R_1~R_{16}$とラベリングされたひずみゲージが貼付されており, ()内に記されたひずみゲージは対となる
ひずみゲージの裏面に貼付されている. ひずみゲージは全て, 中央剛体から3mm離れた位置に配置されている.
ひずみゲージ$R_i$に生じるひずみ${\varepsilon} _i$とひずみ出力$S$は次のような関係で示せる. 
\begin{eqnarray}
  \left\{
    \begin{array}{l}
      S_{F_{x}} = \varepsilon _1 - \varepsilon _2 + \varepsilon _3 -\varepsilon _4 \\
      S_{F_{y}} = \varepsilon _5 - \varepsilon _6 + \varepsilon _7 -\varepsilon _8 \\
      S_{F_{z}} = \varepsilon _9 - \varepsilon _{10} + \varepsilon _{11} -\varepsilon _{12} \\
      S_{M_{x}} = \varepsilon _{13} - \varepsilon _{14} + \varepsilon _{16} -\varepsilon _{15} \\
      S_{M_{x}} = \varepsilon _{11} - \varepsilon _{12} + \varepsilon _{10} -\varepsilon _9 \\
      S_{M_{x}} = \varepsilon _5 - \varepsilon _6 + \varepsilon _8 -\varepsilon _7 \\
    \end{array}
  \right.\\ \nonumber
\end{eqnarray}
\begin{figure}[b]
  \centering
  \subfloat[低剛性起歪体]{\includegraphics[width=6cm]{pic/Low_sim.png}}\\
  \subfloat[高剛性起歪体]{\includegraphics[width=6cm]{pic/High_sim.png}}\\
  \caption[]{有限要素法シミュレーションによる結果}\label{fig:sim}
\end{figure}
\begin{figure}[b]
  \centering
  \subfloat[側面]{\includegraphics[width=4.25cm]{pic/Vertical.png}}
  \subfloat[上面]{\includegraphics[width=4.2cm]{pic/Horizontal.png}}\\
  \caption[]{過負荷防止機構}\label{fig:kahuka}
\end{figure}
\begin{figure}[b]
  \centering
  \subfloat[低剛性起歪体]{\includegraphics[width=4.3cm]{pic/LowGage.png}}
  \subfloat[高剛性起歪体]{\includegraphics[width=4.3cm]{pic/HighGage.png}}\\
  \caption[]{ひずみゲージの貼付位置}\label{fig:gage}
\end{figure}
\subsection{センサ出力}
本力覚センサは印加荷重によって負荷が伝達される起歪体が異なる.
よって低剛性起歪体と高剛性起歪体それぞれから出力信号が取得される.
2つの出力信号には非線形性, 他軸干渉, ヒステリシスといった誤差が含まれており, 
単に起歪体からの出力信号を切り替えるだけではセンサ出力に飛躍が生じる. これによりセンサを導入した
ロボットの制御性能の劣化が生じる危険がある. そこで次式を用い, 
2つの起歪体から得られる出力信号を切り替える. 
\begin{eqnarray}
  \bm{L}_{out} = \alpha \bm{C}^{-1}_L \bm{S}_{L}+\left( 1-\alpha \right)\bm{C}^{-1}_H \bm{S}_{H}
  \label{eq:kirikae}
\end{eqnarray}
ここで$\bm{C}^{-1}$と$\bm{S}$は荷重-ひずみ変換行列とひずみ出力であり,下添え字Lは低剛性, Hは高剛性を意味する. また, $\alpha$は出力信号の仕様比率を表す. 
$\alpha$は次の式で表される. 
\begin{eqnarray}
  \alpha = \left\{
    \begin{array}{ll}
      0 & (L > L^{st}_{th}) \\
      \frac{ L^{fin}_{th} - L }{ L^{fin}_{th} - L^{st}_{th} } & (L^{st}_{th} < L < L^{fin}_{th})\\
      1 & (L < L^{fin}_{th})
      \label{eq:a}
    \end{array}
  \right.
\end{eqnarray}
ここで, $L^{st}_{th}$は切り替え開始の閾値, $L^{fin}_{th}$は切り替え終了の閾値となっている. 
また, 式\eqref{eq:kirikae}, 式\eqref{eq:a}を用いる事で,
\begin{enumerate}
  \item 印加荷重が$L^{st}_{th}$以下の場合:低剛性起歪体の出力のみ使用
  \item 印加荷重が$L^{st}_{th}$以上, $L^{fin}_{th}$以下の場合:飛躍した出力を抑制するため2つの出力を併用
  \item 印加荷重が$L^{fin}_{th}$以上の場合:高剛性起歪体の出力のみを使用
\end{enumerate}
といったようにセンサ出力が決定できる. 

\section{性能試験}
開発したHDR1軸力覚センサの性能を評価するために荷重印加試験を行った.
Fig.~\ref{fig:sensor}に示す荷重印加位置に任意の荷重を印加し, 
その時の金属箔ひずみゲージと半導体ひずみゲージの出力を調べた. 
印加荷重は$0~2.5N$の範囲で$0.5N$刻みに印加した.

実際に製作した力覚センサと実験の様子をFig.~\ref{fig:jissai}示す.
\begin{figure}[h]
  \centering
  \begin{tabular}{c}
    \begin{minipage}{0.5\hsize}
      \begin{center}
        \includegraphics[scale=0.15]{pic/realSensor.jpg}
        \hspace{1.cm} 
        \footnotesize{製作したHDR1軸力覚センサ\\Made HDR unaxial force sensor}
      \end{center}
    \end{minipage}
        \begin{minipage}{0.51\hsize}
      \begin{center}
        \includegraphics[scale=0.03]{pic/jikkenzu.jpg}
        \hspace{1.6cm} 
        \footnotesize{実験の様子\\State of experiment}
      \end{center}
    \end{minipage}
  \end{tabular}
  \caption[]{実際のHDR1軸力覚センサ\\Fig. 5 Actual HDR unaxial force sensor}\label{fig:jissai}
\end{figure}
さらに, 試験より得られた金属箔ひずみゲージの出力をFig.~\ref{fig:kinkeka}, 
半導体ひずみゲージの出力をFig.~\ref{fig:hankekka}に示す. 

Fig.~\ref{fig:kinkeka}, Fig.~\ref{fig:hankekka}を見比べると, 
同じ荷重を印加している時にそれぞれ出力が異なっていることが分かる. 
また, 半導体ひずみゲージの出力のほうが大きく得られていることから, 
低荷重に対し高感度に出力を得られることが確認できた. 

ここで, 1N印加時のシミュレーションによる起歪体自体のひずみ値と 
試験で得られた金属箔, 半導体ひずみゲージの出力を比較する. 
まず, シミューレーションにより1N印加時では
金属箔ひずみゲージ貼り付け位置:$6.658\mu$m/m, 
半導体ひずみゲージ貼り付け位置:$7.047\mu$m/m という結果が得られた. 
さらにFig.~\ref{fig:kinkeka}, Fig.~\ref{fig:hankekka}を見ると, 
金属箔ひずみゲージは無負荷時:$-3.21\mu$, 1N印加時:$8.62\mu$で
半導体ひずみゲージは無負荷時:$1.43\mu$, 1N印加時:$1358\mu$という結果が得られた.
ここで取り上げた数値は印加荷重に変動の無い一定時間の出力を平均したものである. 

金属箔ひずみゲージは無負荷時から1N印加時の間で約11.8$\mu$の変位がみられ,  
シミュレーションで得た起歪体のひずみ出力に対し, 約ゲージ率倍された出力が得られた.
また, 1N印加時の出力をもとに無負荷時のノイズによる出力値を荷重変換すると, 
約0.37Nであった. よって金属箔ひずみゲージでは約0.37N以上の
荷重であればノイズに埋もれることなく取得できる.

半導体ひずみゲージは無負荷時から1N印加時の間で約1357$\mu$の変位がみられる. 
シミュレーションで得た起歪体のひずみ出力に対し, 約ゲージ率倍された出力が得られた.
また, 1N印加時の出力をもとに無負荷時のノイズによる出力値を荷重変換すると, 
約0.001Nであった. よって半導体ひずみゲージでは約0.001N以上の
荷重であればノイズに埋もれることなく取得できる.

この結果より, 1つの起歪体上にゲージ率の異なるひずみゲージを張り付けることで, 
0.001Nから定格荷重である200Nまでの力覚情報の測定が
本センサで可能であることが示された. 

\begin{figure}[h]
  \begin{center}
    \includegraphics[width=8.0cm]{pic/kinkekka.jpg}
    \caption{金属箔ひずみゲージの出力\\Fig. 6 Output of metal foil strain gauge}\label{fig:kinkeka}
  \end{center}
\end{figure}

\begin{figure}[h]
  \begin{center}
    \includegraphics[width=8.0cm]{pic/hankekka.jpg}
    \caption{半導体ひずみゲージの出力\\Fig. 7 Output of semiconductor strain gauge}\label{fig:hankekka}
  \end{center}
\end{figure}

\section{まとめ}
小型かつHDRを有した力覚センサの開発を目的とし, 片持ち梁の起歪体に
金属箔歪ゲージと半導体歪ゲージを貼り付ける構造のセンサを提案した. 

製作したHDR1軸力覚センサの構造は非常に単純な設計で構成することが可能であり, 
半導体歪ゲージで○~○Nの範囲で分解能○の測定精度を, 
金属箔歪ゲージで○~○Nの範囲で分解能○の測定精度を示し, 
SN比より, ○Nから○Nまでの力覚検知が可能であることが示された.  
よって感度の異なる歪ゲージを併用することで, 起歪体の構造自体は単純でも
ダイナミックレンジを有した力覚センサの実現が可能であることが示された. 

今後は1軸のみで実現した本センサを6軸測定が可能なセンサへと拡張を行う. 
これにより小型かつ単純な構造のHDR6軸力覚センサの開発を目指す. 

\section*{謝辞}
この成果は, 国立研究開発法人新エネルギー・産業技術総合開発機構(NEDO)の委託業務の
結果得られたものであることをここに付記し, 関係者各位に謝意を表する. 


\begin{thebibliography}{10}
%\bibliographystyle{plain}    %参考文献出力スタイル(欧文)
%\bibliographystyle{junsrt}    %参考文献出力スタイル(和文)
\bibitem{denso}
力センサ有コンプライアンス機能.
\url{https://www.denso-wave.com/ja/robot/product/function/Acontact.html}
[Online; accessed 2019-02-14] 

\bibitem{asimo}
{Honda|ASIMO}.
\url{https://www.honda.co.jp/ASIMO/}.
[Online; accessed 2019-02-14] 

\bibitem{ROBEAR}
{介護支援ロボット研究用プラットフォームROBEAR}.
\url{http://rtc.nagoya.riken.jp/ROBEAR/}.
[Online; accessed 2019-02-14] 

\bibitem{yoshikawa1989six}
Tsuneo Yoshikawa and Taizou Miyazaki.
\newblock {A six-axis force sensor with three-dimensional cross-shape
  structure}.
\newblock In {\em Proc. IEEE Int. Conf. Robot. Autom.}, pp. 249--255. IEEE,
  1989.

%\bibitem{nishiwaki2002six}
%Koichi Nishiwaki, Yoshifumi Murakami, Satoshi Kagami, Yasuo Kuniyoshi, Masayuki
 % Inaba, and Hirochika Inoue.
%\newblock {A six-axis force sensor with parallel support mechanism to measure
%  the ground reaction force of humanoid robot}.
%\newblock In {\em Proc. IEEE Int. Conf. Robot. Autom.}, Vol.~3, pp. 2277--2282,
%  2002.

%\bibitem{Liang2010}
% Qiaokang Liang, Dan Zhang, Quanjun Song, Yunjian Ge, Huibin Cao, and Yu~Ge.
% \newblock {Design and fabrication of a six-dimensional wrist force / torque
%   sensor based on E-type membranes compared to cross beams}.
% \newblock {\em Measurement}, Vol.~43, pp. 1702--1719, 2010.

\bibitem{Beyeler2009}
Felix Beyeler, Simon Muntwyler, and Bradley~J. Nelson.
\newblock {A six-axis MEMS force-torque sensor with micro-Newton and
  nano-Newtonmeter resolution}.
\newblock {\em J. Microelectromechanical Syst.}, Vol.~18, No.~2, pp. 433--441,
  2009.

\bibitem{Kim2013a}
Ji~Chul Kim, Kyung~Soo Kim, and Soohyun Kim.
\newblock {Note: A compact three-axis optical force/torque sensor using
  photo-interrupters}.
\newblock {\em Rev. Sci. Instrum.}, Vol.~84, No.~12, pp. 77--80, 2013.

%\bibitem{su20093}
%Hao Su and Gregory~S Fischer.
%\newblock {A 3-axis optical force/torque sensor for prostate needle placement
%  in magnetic resonance imaging environments}.
%\newblock In {\em Proc. IEEE Int. Conf. Technol. Pract. Robot Appl.}, pp. 5--9.
%  IEEE, 2009.

%\bibitem{polygerinos2010novel}
%Panagiotis Polygerinos, Pinyo Puangmali, Tobias Schaeffter, Reza Razavi,
%  Lakmal~D Seneviratne, and Kaspar Althoefer.
%\newblock {Novel miniature MRI-compatible fiber-optic force sensor for cardiac
%  catheterization procedures}.
%\newblock In {\em Proc. IEEE Int. Conf. Robot. Autom.}, pp. 2598--2603. IEEE,
%  2010.

\bibitem{Jiang2015}
Jun Jiang, Weihai Chen, Jingmeng Liu, Wenjie Chen, and Jianbin Zhang.
\newblock {Design of a Dual-Range Force Sensor for Achieving High Sensitivity,
  Broad Bandwidth, and Large Measurement Range}.
\newblock {\em IEEE Sens. J.}, Vol.~15, No.~2, pp. 1114--1123, 2015.

\bibitem{jiang2013}
Jun Jiang, Weihai Chen, Jingmeng Liu, and Wenjie Chen.
\newblock {A cost effective multi-axis force sensor for large scale measurement
  : design , modeling , and simulation}.
\newblock pp. 188--193, 2013.

\bibitem{murozaki2014miniaturized}
Yuichi Murozaki, Kousuke Nogawa, and Fumihito Arai.
\newblock {Miniaturized load sensor using quartz crystal resonator constructed
  through microfabrication and bonding}.
\newblock {\em Robomech J.}, Vol.~1, No.~1, p.~3, 2014.

\bibitem{okumura2018high}
Daisuke Okumura, Sho Sakaino, and Toshiaki Tsuji.
\newblock {High Dynamic Range Sensing by a Multistage Six-Axis Force Sensor
  with Stopper Mechanism}.
\newblock In {\em Proc. IEEE Int. Conf. Robot. Autom.}, pp. 1--6. IEEE, 2018.

\bibitem{Okumura}
辻俊明, 奥村大輔, 境野翔.
\newblock 多段型ハイダイナミックレンジ6軸力覚センサの小型化.
\newblock  日本ロボット学会学術講演会, 2018.

\bibitem{1991}
緒方浩二郎, 柏木邦雄, 小野耕三.
\newblock 力覚センサ, 1991.

\bibitem{hanyu2010simplified}
Ryosuke Hanyu, Toshiaki Tsuji, and Shigeru Abe.
\newblock {A simplified whole-body haptic sensing system with multiple
  supporting points}.
\newblock In {\em Proc. IEEE Int. Work. Adv. Motion Control}, pp. 691--696.
  IEEE, 2010.

\bibitem{向井優2018}
向井優, 野田善之.
\newblock 六軸力覚センサの原理と構造.
\newblock 精密工学会誌, Vol.~84, No.~4, pp. 303--306, 2018.


%\bibliography{C:/Users/Kohei/Documents/library}           %data.bibから拡張子を外した名前

%\appendix
  %\input{./src/Refernces}
\end{thebibliography}

%\begin{figure}[b]
  \centering
  \subfloat[$F_x$]{\includegraphics[scale=0.3]{pic/Fx_L.png}}
  \subfloat[$F_y$]{\includegraphics[scale=0.3]{pic/Fy_L.png}}\\
  \subfloat[$F_z$]{\includegraphics[scale=0.3]{pic/Fz_L.png}}
  \subfloat[$M_x$]{\includegraphics[scale=0.3]{pic/Mx_L.png}}\\
  \subfloat[$M_y$]{\includegraphics[scale=0.3]{pic/My_L.png}}
  \subfloat[$M_z$]{\includegraphics[scale=0.3]{pic/Mz_L.png}}\\
  \caption[]{荷重変換後の出力(低剛性起歪体)}\label{fig:kajyuLow}
\end{figure}
\begin{figure}[tb]
  \centering
  \subfloat[$F_x$]{\includegraphics[scale=0.3]{pic/Fx_H.png}}
  \subfloat[$F_y$]{\includegraphics[scale=0.3]{pic/Fy_H.png}}\\
  \subfloat[$F_z$]{\includegraphics[scale=0.3]{pic/Fz_H.png}}
  \subfloat[$M_x$]{\includegraphics[scale=0.3]{pic/Mx_H.png}}\\
  \subfloat[$M_y$]{\includegraphics[scale=0.3]{pic/My_H.png}}
  \subfloat[$M_z$]{\includegraphics[scale=0.3]{pic/Mz_H.png}}\\
  \caption[]{荷重変換後の出力(高剛性起歪体)}\label{fig:kajyuHigh}
\end{figure}
\begin{figure}%[tb]
  \centering
  \subfloat[横軸:線形表示]{\includegraphics[scale=0.325]{pic/SN.png}}
  \subfloat[横軸:対数表示]{\includegraphics[scale=0.325]{pic/SNT.png}}\\
  \caption[]{SN比($F_z)$成分}\label{fig:sn}
\end{figure}
\begin{figure}%[tb]
  \centering
 % \subfloat[低剛性起歪体]{\includegraphics[scale=0.325]{pic/1.png}}
 % \subfloat[高剛性起歪体]{\includegraphics[scale=0.325]{pic/2.png}}\\
  \caption[]{$-F_z$を印加時のひずみ出力}\label{fig:stop}
\end{figure}
%\begin{figure}[tb]
 % \centering
 %\subfloat[$F_x$]{\includegraphics[scale=0.33]{pic/Sx_L.png}}
  %\subfloat[$F_y$]{\includegraphics[scale=0.33]{pic/Sy_L.png}}\\
  %\subfloat[$F_z$]{\includegraphics[scale=0.33]{pic/Sz_L.png}}
  %\subfloat[$M_x$]{\includegraphics[scale=0.33]{pic/SMx_L.png}}\\
  %\subfloat[$M_y$]{\includegraphics[scale=0.33]{pic/SMy_L.png}}
  %\subfloat[$M_z$]{\includegraphics[scale=0.33]{pic/SMz_L.png}}\\
  %\caption[]{非線形性と他軸干渉(低剛性起歪体)}\label{fig:HTLow}
%\end{figure}
%\begin{figure}[tb]
 % \centering
 % \subfloat[$F_x$]{\includegraphics[scale=0.33]{pic/Sx_H.png}}
  %\subfloat[$F_y$]{\includegraphics[scale=0.33]{pic/Sy_H.png}}\\
  %\subfloat[$F_z$]{\includegraphics[scale=0.33]{pic/Sz_H.png}}
  %\subfloat[$M_x$]{\includegraphics[scale=0.33]{pic/SMx_H.png}}\\
  %\subfloat[$M_y$]{\includegraphics[scale=0.33]{pic/SMy_H.png}}
  %\subfloat[$M_z$]{\includegraphics[scale=0.33]{pic/SMz_H.png}}\\
  %\caption[]{非線形性と他軸干渉(高剛性起歪体)}\label{fig:HTHigh}
%\end{figure}

\end{document}