\section{まえがき} 
ロボットは従来, 自動車の溶接や組み立てを行なうといった限定的な作業にのみ従事する産業用途が主であった.
昨今は世界的な労働力不足の対策として細かな部品の組み立てや第三次産業の一端に携わるような
協働ロボットが開発され, よりロボットの活躍の場が広がっている.

協働ロボットの中でも特に注目されているのが, 
人間を支援または人間の代替として活動するロボットである. 
これらのロボットには環境適応力が求められ, 環境情報の取得が必要不可欠となっている.
中でも力覚情報は多様な動作の実現のためには必須だと考えられ, 

実際DENSOの多関節ロボット\cite{denso}の安定した作業やASIMO\cite{asimo}のお茶を注ぐ動作, 
ROBEAR\cite{ROBEAR}の人を持ち上げるといった動作も力覚センサを用い, 
力覚情報を取得することで実現している.

力覚センサは印加された外力によって生じる構造体(以下, 起歪体)の変形量をセンシング素子により検知し,
その変形量を元に印加された外力の推定を行なう. 
センシング素子により力覚センサの測定方式は異なり, 
ひずみゲージ式\cite{yoshikawa1989six}%\cite{nishiwaki2002six}\cite{Liang2010}, 
静電容量式\cite{Beyeler2009}, 
光学式\cite{Kim2013a}%\cite{su20093}\cite{polygerinos2010novel}
など様々な検出方法が実用化されている. 
力覚センサの測定レンジは起歪体の剛性により決まり, 
低剛性の起歪体は小さな力の検出が, 高剛性の起歪体は大きな力の検出が可能である. 
しかしこれは反対に, 1つの起歪体だけでは小さな力と大きな力の検出を行なえないことを示唆している. 
起歪体の変形を検知するという原理上, 分解能と定格荷重はトレードオフの関係にあり, 従来の力覚センサの
ダイナミックレンジ(測定可能な力の最小値と最大値の比)には制限が生じる. 

これに対しJiangらは低剛性起歪体と高剛性起歪体を組み合わせた1軸\cite{Jiang2015}, 2軸\cite{jiang2013}
の力情報を検知可能なハイダイナミックレンジ(HDR:High Dynamic Range)力覚センサを提案した. 
また起歪体を用いず, 水晶振動子による圧電効果を利用しHDRで力覚検知を可能にした
1軸のセンサも提案されている\cite{murozaki2014miniaturized}. 
さらに1軸, 2軸での検知のみであったHDR力覚センサに対し, Okumuraらは6軸の力情報を取得可能な
HDR力覚センサ(size:150×150×45 mm)を提案した\cite{okumura2018high}. これはそれぞれが6軸の力情報を取得できる
低剛性起歪体と高剛性起歪体を二段に重ね合わせた構造をしており, 
0.01Nから1000Nまでの力覚検知が可能である. 従来の10倍以上のダイナミックレンジを
有した力覚センサとなった. 

しかし, 提案されてきたHDR6軸力覚センサはセンサ自体のサイズが大きく, 
先述したようなロボットに導入する上で大きな問題となる.
この問題に対し, Okumuraらはさらに, 低剛性起歪体と高剛性起歪体を一層に集約した樹脂製のHDR6軸力覚センサ(size:100×100×30 mm)を
提案した\cite{Okumura}. 樹脂製HDR6軸力覚センサは起歪体を水平に配置することで 
センサの高さが低くなる. また高剛性起歪体の梁を短く設計したことで体積が小さくなった. 
しかし, 樹脂製HDR6軸力覚センサは素材の特性上, クリープ現象や応力緩和といった金属製センサではあまり見られなかった
特性が生じた. このことからHDR6軸力覚センサを小型化する上で, センサ構造自体を工夫した金属製の力覚センサを
提案することが重要だと考えられる. 

そこで本論文では, 小型なHDR6軸力覚センサの開発を目的とし, 
新規な構造であるクロスアーチ型の低剛性起歪体を導入したHDR6軸力覚センサを提案する. 
ここで提案する新規構造はアーチ状に設計した起歪体を交差させた形状のものであり, 
本構造をクロスアーチ型と呼ぶ. 

%以下に本論文の構成を示す. まず,2章で力覚センサに関連する原理を述べる.
%次に3章では提案する小型HDR6軸力覚センサを示し,4章でシミュレーションによって, 提案する力覚センサの
%挙動と設計の妥当性を確認する. 5章では実際に製作した力覚センサの有用性を確認するため行なった性能試験の結果を述べ, 
%最後の6章でまとめとする. 