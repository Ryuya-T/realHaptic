\section{まえがき} 
ロボットの活躍領域は近年急速に拡大しており, 
それとともに普及率も上昇している. 
従来は自動車や家電など製造を行う産業用途での活躍が主であったが, 
現在は医療, 介護やサービス業などの第三次産業に有効性が見込まれ, 
協働ロボット等が多く開発されていることからもその背景がうかがい知れる.

協働ロボットの中でも特に人間を支援するまたは人間の代替として活動するロボットが注目されている. 
これらのロボットには環境適応力が求められ, 環境情報の取得が必要不可欠となっている.
中でも力覚情報は多様な動作の実現のためには必須だと考えられ, 
実際ロットによる多くの作業で力覚センサが用いられている\cite{blumenkranz2018force}\cite{Berkman2003}\cite{Bouyarmane2019}\cite{Jacq2019}\cite{yokoo2012}.

これまでの力覚センサは, 発生する力の大きさに合わせて, 測定に際し
適切なレンジの力覚センサを選出し使用されてきた. 
しかし多種多様な動作を一台のロボットで実現させる場合, それに導入する力覚センサは
レンジが限定的なものではなく, よりハイダイナミックレンジ(HDR:High Dynamic Range)で
あることが求められる. 
また, HDRなだけではなく, ロボットに取り付け可能なサイズであることが求められる. 

このような需要に対し, Jiangらは低剛性起歪体と高剛性起歪体を組み合わせた1軸\cite{Jiang2015}, 2軸\cite{jiang2013}
の力情報を検知可能なHDR力覚センサを提案した. 
また起歪体を用いず, 水晶振動子による圧電効果を利用しHDRで力覚検知を可能にした
1軸のセンサも提案されている\cite{murozaki2014miniaturized}.

さらに1軸, 2軸での検知のみであったHDR力覚センサに対し, Okumuraらは6軸の力情報を取得可能な
HDR力覚センサ(size:150×150×45 mm)を提案した\cite{okumura2018high} \cite{Okumura}. 
これはそれぞれが6軸の力情報を取得できる低剛性起歪体と高剛性起歪体を二段に重ね合わせた構造をしており, 
0.01Nから1000Nまでの力覚検知が可能である. 従来の10倍以上のダイナミックレンジを
有した力覚センサとなった. 

しかし, 提案されてきたHDR力覚センサはセンサ自体のサイズが大きく, 
先述したようなロボットに導入する上で大きな問題となる.

これに対し我々は起歪体の構造自体を工夫することでHDRと小型化の両立を図るセンサ\cite{Ryuya}の提案をした. 
このセンサは低剛性起歪体の形状がクロスアーチ型となっており, 
高剛性起歪体の隙間にフィットする構造となっている. 
センササイズは(80×80×23.5 mm)と小型化に成功し, 
測定レンジは0.2N~500Nを確保出来た.
しかし, 0.2N以下の低荷重域での測定が不向きであることや, 
従来センサのHDRの広さを保つことが出来なかった. 

これらの研究ではストッパ機構を有する多段型のメカニズムを小型化すると
非線形性が増大することが判明した. この課題を解決するためには単純な
構造の起歪体でダイナミックレンジを拡張する方法を考える必要がある. 

そこで本論文では, HDRを実現する新たなる手法として, 半導体ひずみゲージと
金属箔ひずみゲージを併用した小型で単純な構造の力覚センサを提案する. 
従来は剛性の異なる起歪体を多段に使用することで実現していたHDRを, 
本センサでは単純な起歪体構造(片持ち梁)上に感度の異なるひずみゲージを導入することでHDRを実現した. 