\section{まえがき} 

ロボットの活躍領域は近年急速に拡大しており, 
それとともに普及率も指数関数的に上昇している. 
この背景は, 従来は自動車や家電など製造を行う産業用途での活躍が主であったが, 
現在は医療, 介護やサービス業などの第三次産業に有効性が見込まれ, 
協働ロボット等が多く開発されていることからもわかる.

協働ロボットの中でも特に人間を支援または人間の代替として活動するロボットが注目されている. 
これらのロボットには環境適応力が求められ, 環境情報の取得が必要不可欠となっている.
中でも力覚情報は多様な動作の実現のためには必須だと考えられ, 
実際DENSOの多関節ロボット\cite{denso}の安定した作業やASIMO\cite{asimo}のお茶を注ぐ動作, 
ROBEAR\cite{ROBEAR}の人を持ち上げるといった動作も力覚センサを用い, 
力覚情報を取得することで実現している.

これまでの力覚センサは, 発生する力の大きさに合わせて, 測定に際し
適切なレンジの力覚センサを選出し使用されてきた. 
しかし多種多様な動作を一台のロボットで実現させる場合, それに導入する力覚センサは
レンジが限定的なものではなく, よりハイダイナミックレンジ(HDR:High Dynamic Range)で
あることが求められる. 
また, HDRなだけではなく, ロボットに取り付け可能なサイズであることが求められる. 

このような需要に対し, Jiangらは低剛性起歪体と高剛性起歪体を組み合わせた1軸\cite{Jiang2015}, 2軸\cite{jiang2013}
の力情報を検知可能なHDR力覚センサを提案した. 
また起歪体を用いず, 水晶振動子による圧電効果を利用しHDRで力覚検知を可能にした
1軸のセンサも提案されている\cite{murozaki2014miniaturized}.

さらに1軸, 2軸での検知のみであったHDR力覚センサに対し, Okumuraらは6軸の力情報を取得可能な
HDR力覚センサ(size:150×150×45 mm)を提案した\cite{okumura2018high}. 
これはそれぞれが6軸の力情報を取得できる低剛性起歪体と高剛性起歪体を二段に重ね合わせた構造をしており, 
0.01Nから1000Nまでの力覚検知が可能である. 従来の10倍以上のダイナミックレンジを
有した力覚センサとなった. 

%力覚センサは外力によって生じる構造体(以下, 起歪体)の変形量をセンシング素子により検知し,
%その変形量を元に外力の推定を行なう. 
%センシング素子により力覚センサの測定方式は異なり, 
%ひずみゲージ式\cite{yoshikawa1989six}%\cite{nishiwaki2002six}\cite{Liang2010}, 
%静電容量式\cite{Beyeler2009}, 
%光学式\cite{Kim2013a}%\cite{su20093}\cite{polygerinos2010novel}
%など様々な検出方法が実用化されている. 

%基本的に力覚センサの測定レンジは起歪体の剛性により決まり, 
%低剛性の起歪体は小さな力の検出が, 高剛性の起歪体は大きな力の検出が可能である. 
%しかしこれは反対に, 1つの起歪体だけでは小さな力と大きな力の検出を行なえないことを示唆している. 
%起歪体の変形を検知するという原理上, 分解能と定格荷重はトレードオフの関係にあり, 従来の力覚センサの
%ダイナミックレンジ(測定可能な力の最小値と最大値の比)には制限が生じる. 

しかし, 提案されてきたHDR力覚センサはセンサ自体のサイズが大きく, 
先述したようなロボットに導入する上で大きな問題となる.

これに対し我々は起歪体の構造自体を工夫することでHDRと小型化の両立を図るセンサの提案をした. 
このセンサは低剛性起歪体の形状がクロスアーチ型となっており, 
高剛性起歪体の隙間にフィットする構造となっている. センササイズは〇と小型化に成功し, 
測定レンジは〇を確保出来た.
しかし, 低荷重域での測定が不向きであることや, 
従来センサのHDRの広さを保つことが出来なかった. 

%この問題に対し, Okumuraらはさらに, 低剛性起歪体と高剛性起歪体を一層に集約した
%樹脂製のHDR6軸力覚センサ(size:100×100×30 mm)を
%提案した\cite{Okumura}. 樹脂製HDR6軸力覚センサは起歪体を水平に配置することで 
%センサの高さが低くなる. また高剛性起歪体の梁を短く設計したことで体積が小さくなった. 
%しかし, 樹脂製HDR6軸力覚センサは素材の特性上, クリープ現象や応力緩和といった金属製センサではあまり見られなかった
%特性が生じた. 

このことから, 小型かつHDRを有した力覚センサの実現には, 
被測定材料にあたるセンサ構造自体を工夫するのではなく, 
センシング部分の改革が必要であると考えた. 

そこで本論文では, HDRを実現する新たなる手法として, 半導体歪ゲージと金属箔歪ゲージを併用した
小型で単純な構造の力覚センサを提案する. 
従来は剛性の異なる起歪体を多段に使用することで実現していたHDRを, 
本センサでは単純な起歪体構造(片持ち梁)上に感度の異なるひずみゲージを導入すること(のみ)でHDRを実現した. 

%以下に本論文の構成を示す. まず,2章で力覚センサに関連する原理を述べる.
%次に3章では提案する小型HDR6軸力覚センサを示し,4章でシミュレーションによって, 提案する力覚センサの
%挙動と設計の妥当性を確認する. 5章では実際に製作した力覚センサの有用性を確認するため行なった性能試験の結果を述べ, 
%最後の6章でまとめとする. 