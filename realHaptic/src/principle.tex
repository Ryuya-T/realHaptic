\section{力覚センサの原理}
力覚センサは力覚情報を測定するセンサである. 
力覚情報を測定するためには, 被測定材料のひずみ, 変化量, または素子の特性変化等を電気信号に
変換する必要がある. 
実際, 力覚センサには様々な測定方式が採用されており, 
ひずみゲージ式\cite{yoshikawa1989six}%\cite{nishiwaki2002six}\cite{Liang2010}, 
静電容量式\cite{Beyeler2009}, 
光学式\cite{Kim2013a}%\cite{su20093}\cite{polygerinos2010novel}
などが実用化されている.
本研究で提案する力覚センサはこの中でもひずみゲージ式の力覚センサである. 

\subsection{ひずみゲージ}
材料に引張力(または圧縮力)$P$が加わる時, これに対応する応力$σ$が材料内部に発生する.
ここでこの応力に比例した引張ひずみ(または圧縮ひずみ)が発生し, 長さ$L$の材料は
$L + \Delta L$(または$L - \Delta L$)に変形する. 
この時の$L$と$\Delta L$の割合をひずみと呼ぶ. 

ひずみゲージはこのひずみを電気信号として検出することのできるセンシング素子のことである. 

本研究で使用するひずみゲージは金属箔ひずみゲージと半導体ひずみゲージの二種類である. 
ひずみゲージは被測定対象に貼り付けて使用される. 
測定対象に力が加わりひずみが発生すると張り付けたひずみゲージにひずみが伝達される. 
この時ひずみゲージはひずみの大きさに比例して内部の抵抗体の抵抗値が変化する. 
これを検出し印加荷重と結び付けることで力覚情報の取得が行える. 

式1

式1は発生したひずみとひずみゲージの抵抗変化の関係式である. 
各変数は $:ひずみ, R:抵抗, R:抵抗変化, K:ゲージ率$である. 
ここで$K$はゲージ率と呼ばれるもので, 各ひずみゲージ固有の比例定数である. 
またゲージ率の大きさは感度の高さを表す指標となる. 

表1に本研究で使用したひずみゲージの詳細を示す. 
使用するひずみゲージのゲージ率はそれぞれ異なる値となっている. 
このゲージの感度の違いを利用し, 半導体ひずみゲージでは低荷重域を, 
金属箔ひずみゲージでは高荷重域のひずみの検出を行い, 
組み合わせることでHDRを実現する. 

\subsection{起歪体}

%ひずみ出力\textbf{\textit{S}}[$\mu$m/m]は次のように表せる. 
%\begin{eqnarray}
%  \bm{L} = {[F_x, F_y, F_z, M_x, M_y, M_z]}^{\top} \\
%  \bm{S} = {[ S_1, S_2, S_3, S_4, S_5, S_6 ]}^{\top}
%\end{eqnarray}

